\begin{figure}[h]
\begin{center}
%\includegraphics*[angle=270, width=11cm]{beispielbild}
\begin{tikzpicture}
\begin{axis}[
	%title=Winkelrichtgröße bestimmen,
	xlabel={Auslenkung [$^\circ$]},
	ylabel={Kraft [N]},
	grid=major,
	legend style={at={(0,1)},anchor=north west}
]

\addplot [no markers, red] gnuplot [raw gnuplot] {
	f(x) = a + b * x; % Fit function
	a = 0.01; b = 0.002; % Set reasonable starting values here
	set fit errorvariables;
	fit f(x) 'A1.txt' via a, b;
	plot [x=-180:180] f(x);
	set print 'parameters.dat'; % Open a file to save the parameters into
	print a, a_err; % Write the parameters to file
	print b, b_err;
};

\addplot [
	blue,
	only marks,
	mark = |,
	mark size = 1.5,
	error bars/.cd,
	y dir = both,
	y fixed = 0.03,
	x dir = both,
	x fixed = 2 % 2° Abweichung
] table {A1.txt};

% We compute our own error values with a given o_y
%\addlegendentry{
%	\pgfplotstableread{parameters.dat}\parameters % Open the file Gnuplot wrote
%	\pgfplotstablegetelem{0}{0}\of\parameters \pgfmathsetmacro\paramA{\pgfplotsretval} % Get first element, save into \paramA
%	\pgfplotstablegetelem{1}{0}\of\parameters \pgfmathsetmacro\paramB{\pgfplotsretval}
%	$\pgfmathprintnumber{\paramB} \cdot \varphi
%		+ \pgfmathprintnumber{\paramA}$
%}

\end{axis}
\end{tikzpicture}
\caption{Winkelrichtgröße des kleinen Drehtellers bestimmen}\label{dia:A1}
\end{center}
\end{figure}
