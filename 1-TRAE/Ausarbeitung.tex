%&PDFLaTeX
\documentclass[a4paper, 11pt, ngerman]{scrartcl}
\usepackage[utf8]{inputenc}
\usepackage[ngerman]{babel}
\usepackage[T1]{fontenc}
\usepackage{fancyhdr}
\usepackage{graphicx}
\usepackage{pgfplots}
\usepackage{pgfplotstable}
\usepackage{hyperref}
\usepackage{cleveref}
\usepackage{hyphenat}
\usepackage[separate-uncertainty=true]{siunitx}

\hypersetup{pdftitle={Trägheitsmoment}, pdfauthor={Daniel Schubert, Sebastian Neubauer}, pdfsubject={Physikpraktikum}}
\pgfplotsset{width=13cm, height=7cm, compat=1.12}

\setlength{\parindent}{0ex}
\setlength{\parskip}{1.5ex plus0.5ex minus0.2ex}
\setlength{\headheight}{26pt}
\addtolength{\topmargin}{-0.2in}
\addtolength{\headsep}{0.2in}
\addtolength{\oddsidemargin}{-0.2in}
\addtolength{\textwidth}{0.5in}

\renewcommand{\topfraction}{1.0}
\renewcommand{\bottomfraction}{1.0}
\renewcommand{\textfraction}{0.0}

% Allow centered columns with fixed width
\newcolumntype{C}[1]{>{\centering\arraybackslash}p{#1}}

\hyphenation{Träg-heits-mo-men-te Träg-heits-mo-men-tes}

\newcommand{\stand}{20.03.16}
\pagestyle{fancy}
\rhead{\textbf{Stand: \stand \\ Seite \thepage}}
\lhead{\textbf{Mechanik\\Trägheitsmoment}}
\chead{\textbf{(TRAE)}}
\rfoot{}
\lfoot{}
\cfoot{}

\begin{document}
\thispagestyle{empty}
\begin{center}
% Prinzipiell kann LaTeX auch Titelseiten erzeugen, ich habe mich
% hier trotzdem für die händische Variante entschieden.
\textbf{\Large Trägheitsmoment (TRAE)}\\[1ex]
\textbf{\large Themengebiet Mechanik}\\[5ex]
       Autoren: Daniel Schubert und Sebastian Neubauer\\
       Gruppe 1, Team 16, Nummer 231\\
       Technische Universität München\\[1ex]
       \stand
\end{center}

\tableofcontents
\clearpage

\section{Einleitung}\label{sec:einleitung}
Der Versuch zum Trägheitsmoment bestimmt das Trägheitsmoment einer Puppe und das eines Menschen in verschiedenen Haltungen auf einem Drehteller. Dazu werden die Eigenschaften der Drehscheiben auf verschiedene Weisen berechnet.

Es soll unter anderem das Drehmoment einer Drehscheibe mit Hilfe eines Federkraftmessers bestimmt werden. Der Federkraftmesser sollte dazu im $90^\circ$ Winkel zu der Stange der Drehscheibe stehen. Für einen maximalen Fehler von $1\%$ gilt für die Abweichung $\Theta$: $cos(\Theta) \ge \frac{F}{1.01 \cdot F}$ und damit $\Theta \le acos\left(1 \over 1.01\right) \approx 8.1^\circ$

Wenn man das Trägheitsmoment der Puppe, bzw. des Menschen, bestimmt, muss beachtet werden, dass sich der Schwerpunkt nicht allzu weit von der Drehachse entfernt, weil sonst der Satz von Steiner beachtet werden muss. Man kann für diesen Fall die Person als Zylinder mit $J_Z = \frac{mR^2}{2}$ approximieren. Um den Fehler unter $10\%$ zu halten gilt: $md^2 < 0.1 \cdot \frac{mR^2}{2} \Rightarrow d < \sqrt{0.05 R^2} \approx 0.22 \cdot R$ Der Schwerpunkt eines Menschen darf also nur ca. $\SI{3}{cm}$ von der Drehachse abweichen (bei einem Radius des Zylinders von $\SI{15}{cm}$), die Puppe um $\SI{4.5}{mm}$ ($R = \SI{2}{cm})$.

\section{Verwendete Methoden}
\subsection{Eigenschaften der Puppe}
Es soll das Trägheitsmoment der Puppe bestimmt werden. Dazu werden zuerst die Eigenschaften des kleinen Drehtellers berechnet. Die Winkelrichtgröße $D^*_k$ kann über das rückstellende Drehmoment statisch bestimmt werden:
\begin{equation}\label{equ:Drehmoment}
F \cdot r_k = M = -D^*_k \cdot \varphi
\end{equation}

Um das Trägheitsmoment des Drehtellers selbst zu bestimmen wird ein Gewichtepaar angehängt und die Schwingungsdauer gemessen. Über $T = 2\pi \cdot \sqrt{J_\omega \over D^*}$ und das Trägheitsmoment der Gewichte
\begin{equation}\label{equ:TraegheitsmomentZylinderBot}
J_{Z\bot} = m \cdot \left(\frac{R_Z^2}{4} + \frac{h_Z^2}{12}\right)
\end{equation}
kann das Trägheitsmoment des Drehtellers $J_k$ berechnet werden:
\begin{equation}\label{equ:Traegheitsmoment}
T^2 = \frac{4\pi^2}{D^*_k} \cdot (J_k + J_Z)
\end{equation}

Für die Messung des Trägheitsmoments der Puppe wird die Puppe in mehreren Posen auf den Drehteller gesteckt und die Schwingungsdauer gemessen.

Anschließend muss noch die Puppe selbst vermessen werden, damit sie mit dem Menschen verglichen werden kann. Dabei werden Masse $m_p$, Größe $h_p$, Hüftumfang $u_p$, Armlänge $l_{ap}$, Armumfang $u_{ap}$, Beinlänge $l_{bp}$, Beinumfang $u_{bp}$ und Kopfumfang $u_{kp}$ gemessen.

\subsection{Eigenschaften des Menschen}
Analog zu Puppe, werden die Größen des großen Drehtellers für den Menschen bestimmt ($D^*_g$, $J_g$). Zusätzlich wird das Eigenträgheitsmoment des Drehtellers noch abgeschätzt durch einen Zylinder:
\begin{equation}\label{equ:TraegheitsmomentZylinder}
J_g = \frac{\overbrace{\pi h_g R_g^2 \cdot \rho}^m \cdot  R_g^2}{2}
\end{equation}
Die Masse wird durch das Volumen und die gegebene Dichte geschätzt, dafür müssen die Höhe $h_g$ und der Radius $R_g$ gemessen werden.

Auch hier wird die Schwingungsdauer $T$ gemessen, wenn der Mensch in verschiedenen Posen auf dem Drehteller steht. Masse $m_m$, Größe $h_m$, Hüftumfang $u_m$, Armlänge $l_{am}$, Armumfang $u_{am}$, Beinlänge $l_{bm}$, Beinumfang $u_{bm}$ und Kopfumfang $u_{km}$ des Menschen müssen auch erfasst werden.

\section{Experimentelles Vorgehen}
Sowohl bei den Gewichten als auch beim großen Drehteller wurde die Ressourcen aus Nummer 2 verwendet. Beim großen Drehteller wurde für die Bestimmung des Gewichts bzw. des Eigenträgheitsmoments die Haltestange vernachlässigt.

Die Kräfte bei Auslenkungen des kleinen Drehtellers wurden mit einem Federkraftmesser gemessen, dessen Genauigkeit bei ca. $\SI{0.03}{N}$ liegt, weil die Haftung im Messgerät sehr hoch war und sich deshalb bei geringen Winkeländerungen das Ergebnis nicht geändert hat. Die Messgenauigkeit des stärkeren Federkraftmessers, der für den großen Teller verwendet wurde, liegt bei $\SI{0.2}{N}$.

Für die meisten Längenmessungen wurde ein Maßband verwendet, dabei schätzen wir die Ungenauigkeit auf $\SI{2}{mm}$. Nur für die Abmessung der Gewichte und die Dicke des großen Drehtellers wurde ein Messschieber verwendet, dessen Genauigkeit wir auf $\SI{1}{mm}$ schätzen. Bei den zylinderförmigen Gewichten wurden die Schrauben zum Festmachen vernachlässigt.

Für die Messung von kleinen Gewichten wurde eine Küchenwaage mit einer Genauigkeit von $\SI{1}{g}$ verwendet. Die Menschen wurden mit einer anderen Waage gewogen, deren Genauigkeit bei ca. $\SI{1}{kg}$ liegt.

Bei Zeitmessungen für den kleinen Drehteller wurde eine Lichtschranken-Uhr verwendet, deren Genauigkeit wir auf $\SI{1}{ms}$ schätzen, weil bei mehreren Messungen kein System in weiteren Ziffern erkennbar war. Für den großen Teller, wurde eine Stoppuhr und ein Computer mit Spannungsmessgerät, das mit einem Drehwiderstand verbunden war verwendet. Die Genauigkeit der Zeitmessungen mit der Stoppuhr liegt bei ca. $\SI{0.1}{s}$ bedingt durch die Schwankung der Reaktionszeit (sowohl Beginn als auch Ende der Messung sind um die Reaktionszeit verschoben).

Die Auslenkungswinkel wurden mit Beschriftungen und dem Auge abgeschätzt, sodass sich eine Ungenauigkeit von $\SI{2}{^\circ}$ ergibt.

\section{Ergebnisse}
\subsection{Aufgabe 10}
\begin{figure}[h]
\begin{center}
%\includegraphics*[angle=270, width=11cm]{beispielbild}
\begin{tikzpicture}
\begin{axis}[
	%title=Winkelrichtgröße bestimmen,
	xlabel={Auslenkung [$^\circ$]},
	ylabel={Kraft [N]},
	grid=major,
	legend style={at={(0,1)},anchor=north west}
]

\addplot [no markers, red] gnuplot [raw gnuplot] {
	f(x) = a + b * x; % Fit function
	a = 0.01; b = 0.002; % Set reasonable starting values here
	set fit errorvariables;
	fit f(x) 'A1.txt' via a, b;
	plot [x=-180:180] f(x);
	set print 'parameters.dat'; % Open a file to save the parameters into
	print a, a_err; % Write the parameters to file
	print b, b_err;
};

\addplot [
	blue,
	only marks,
	mark = |,
	mark size = 1.5,
	error bars/.cd,
	y dir = both,
	y fixed = 0.03,
	x dir = both,
	x fixed = 2 % 2° Abweichung
] table {A1.txt};

% We compute our own error values with a given o_y
%\addlegendentry{
%	\pgfplotstableread{parameters.dat}\parameters % Open the file Gnuplot wrote
%	\pgfplotstablegetelem{0}{0}\of\parameters \pgfmathsetmacro\paramA{\pgfplotsretval} % Get first element, save into \paramA
%	\pgfplotstablegetelem{1}{0}\of\parameters \pgfmathsetmacro\paramB{\pgfplotsretval}
%	$\pgfmathprintnumber{\paramB} \cdot \varphi
%		+ \pgfmathprintnumber{\paramA}$
%}

\end{axis}
\end{tikzpicture}
\caption{Winkelrichtgröße des kleinen Drehtellers bestimmen}\label{dia:A1}
\end{center}
\end{figure}

Die Ausgleichsgerade für die Kraftmessung in Abhängigkeit von der Auslenkung des kleinen Drehtellers hat folgende Gleichung:
\begin{equation}
F = \SI{0.004 \pm 0.011}{N} + \underbrace{\SI{2.17 \pm 0.09}{10^{-3}N}}_{\text{Steigung } a_1} \cdot \frac{180}{\pi} \cdot \varphi
\end{equation}
Wie leicht erkannt werden kann, liegt der Ursprung innerhalb des Fehlerbereichs der Geraden. $D^*_k$ kann durch \cref{equ:Drehmoment} berechnet werden, indem nach $D^*_k$ umgestellt wird: $D^*_k = -r_k \cdot a_1 \cdot \frac{180}{\pi} = \SI{0.0236 \pm 0.0010}{Nm}$\\
Dabei wurde die Unsicherheit der Geradensteigung und des Radius miteinberechnet.

\subsection{Aufgabe 11}
Um das Eigenträgheitsmoment des Tellers zu bestimmen, wurden Gewichte in unterschiedlichen Abständen angehängt.
Die Zylinder haben die folgenden Maße: $m = \SI{49}{g}$, $h = \SI{8}{mm}$, $d = \SI{3.0}{cm}$
\begin{table}[ht]
\begin{center}\label{tab:Gewichte}
\begin{tabular}{r|ccccc}\hline
\textbf{Abstand} ($\SI{\pm 0.2}{cm}$) [cm] & $0$ & $8$ & $11$ & $15$ & $19$ \\
\hline
\textbf{Trägheitsmoment $J_Z$} [$10^{-3}$kgm$^2$] & $0$ & $0.317$ & $0.596$ & $1.106$ & $1.772$ \\
\textbf{$md^2$} [$10^{-3}$kgm$^2$]                & $0$ & $0.314$ & $0.593$ & $1.103$ & $1.769$ \\
\hline\end{tabular}
\caption{Trägheitsmomente der angehängten Gewichte}
\end{center}
\end{table}

$\Delta (md^2) = \Delta d \cdot 2md \Rightarrow \SI{0.049}{kg} \cdot (\SI{0.11}{m})^2 = \SI{0.593 \pm 0.022}{10^{-3} kgm^2}$\\
Es kann festgestellt werden, dass die Ungenauigkeit des Abstands $r$ einen deutlich größeren Einfluss auf das Trägheitsmoment hat als wenn der Zylinder als Punktmasse betrachtet wird.
\begin{figure}[ht]
\begin{center}
%\includegraphics*[angle=270, width=11cm]{beispielbild}
\begin{tikzpicture}
\begin{axis}[
	%title=Winkelrichtgröße bestimmen,
	xlabel={$2md^2$ [kgm$^2$]},
	ylabel={$T^2$ [s$^2$]},
	grid=major,
	legend style={at={(0,1)},anchor=north west}
]

\addplot [no markers, green] gnuplot [raw gnuplot] {
	f(x) = a + b * x; % Fit function
	a = 0.01; b = 2000; % Set reasonable starting values here
	set fit errorvariables;
	fit f(x) '< cat A2-90.txt A2-135.txt | sort -n' using ($1 * $1 * 0.049 * 2):($2 * $2) via a, b;
	plot [x=-0.0003:0.004] f(x);
	set print 'parameters.dat'; % Open a file to save the parameters into
	print a, a_err; % Write the parameters to file
	print b, b_err;
};

\addlegendentry{
	\pgfplotstableread{parameters.dat}\parameters % Open the file Gnuplot wrote
	\pgfplotstablegetelem{0}{0}\of\parameters \pgfmathsetmacro\paramA{\pgfplotsretval} % Get first element, save into \paramA
	\pgfplotstablegetelem{1}{0}\of\parameters \pgfmathsetmacro\paramB{\pgfplotsretval}
	$\pgfmathprintnumber{\paramB} \cdot 2md^2
		+ \pgfmathprintnumber{\paramA}$
}

\addplot [
	red,
	only marks,
	mark = |,
	mark size = 2,
	error bars/.cd,
	y dir = both,
	y explicit,
	x dir = both,
	x explicit
] table [x expr = {\thisrowno{0} * \thisrowno{0} * 0.049 * 2}, y expr = {\thisrowno{1} * \thisrowno{1}},
	% Estimated error for T: 0.01s
	y error expr = {sqrt(2) * 0.01 * \thisrowno{1}},
	% r: 0.002m, m: 0.001kg
	x error expr = {sqrt(0.001 ^ 2 * 4 * \thisrowno{0} ^ 4 + 0.002 ^ 2 * 16 * (0.049 * \thisrowno{0}) ^ 2)}
	] {A2-90.txt};

\addlegendentry{$90^\circ$}

\addplot [
	blue,
	only marks,
	mark = |,
	mark size = 2,
	error bars/.cd,
	y dir = both,
	y explicit,
	x dir = both,
	x explicit
] table [x expr = {\thisrowno{0} * \thisrowno{0} * 0.049 * 2}, y expr = {\thisrowno{1} * \thisrowno{1}},
	% Estimated error for T: 0.01s
	y error expr = {sqrt(2) * 0.01 * \thisrowno{1}},
	% r: 0.002m, m: 0.001kg
	x error expr = {sqrt(0.001 ^ 2 * 4 * \thisrowno{0} ^ 4 + 0.002 ^ 2 * 16 * (0.049 * \thisrowno{0}) ^ 2)}
	] {A2-135.txt};

\addlegendentry{$135^\circ$}

\end{axis}
\end{tikzpicture}
\caption{Trägheitsmoment des kleinen Drehtellers bestimmen}\label{dia:A2}
\end{center}
\end{figure}

Die Ausgleichsgerade für $T^2$ in Abhängigkeit von $2md^2$ hat folgende Gleichung:
\begin{equation}
T^2 = \underbrace{\SI{1697 \pm 21}{1 \over Nm}}_{a_2} \cdot 2md^2 + \underbrace{\SI{0.46 \pm 0.04}{s^2}}_{a_3}
\end{equation}

Aus Gleichung (18) in der Angabe folgt
\begin{equation}
J_k = \frac{a_3}{a_2}
\end{equation}
und
\begin{equation}
D^*_k = \frac{4\pi^2}{a_2}.
\end{equation}
Damit können $D^*_k = \SI{0.02326 \pm 0.00028}{Nm}$ und $J_k = \SI{2.71 \pm 0.24}{10^{-4}kgm^2}$ berechnet werden. Da $J_k$ auch von $a_3$ abhängt, muss hier noch diese Unsicherheit miteinberechnet werden. In den Unsicherheiten von $J_k$ ist auch die Unsicherheit der Masse enthalten. Der jetzt ausgerechnete Wert von $D^*_k$ ist mehr als $3\times$ so genau wie der vorherige Wert (liegt aber im Unsicherheitsbereich des vorherigen Wertes), deshalb sollte mit diesem Wert weitergerechnet werden.

\subsection{Aufgabe 12}\label{sub:A12}
Um das Trägheitsmoment der Puppe in den verschiedenen Positionen ($J_{p1}$ mit angelegten Armen und $J_{p2}$ mit ausgestreckten Armen) zu berechnen, kann auch Gleichung (18) aus der Aufgabenstellung zu Hilfe genommen werden. Damit kommt man auf folgende Gleichung ($T_0$ ist die Schwingungsdauer ohne Gewichte und ohne Puppe)
\begin{equation}\label{equ:TraegheitsmomentPuppe}
J_p = J_k \cdot \left(\frac{T_p^2}{T_0^2} - 1\right)
\end{equation}
und $J_{p1} = \SI{7.4 \pm 0.8}{10^{-5}kgm^2}$, bzw. $J_{p2} = \SI{28.5 \pm 1.2}{10^{-5}kgm^2}$. Der Fehler wird auch hier mit quadratischer Addition berechnet, dabei wurden die Unsicherheiten von $J_k$, $T_p$ und $T_0$ beachtet. Das Verhältnis der Trägheitsmomente ist damit ca. $\frac{J_{p2}}{J_{p1}} = 3.9$.

\subsection{Aufgabe 13}
\begin{figure}[h]
\begin{center}
%\includegraphics*[angle=270, width=11cm]{beispielbild}
\begin{tikzpicture}
\begin{axis}[
	%title=Winkelrichtgröße bestimmen,
	xlabel={Auslenkung [$^\circ$]},
	ylabel={Kraft [N]},
	grid=major,
	legend style={at={(0,1)},anchor=north west}
]

\addplot [no markers, red] gnuplot [raw gnuplot] {
	f(x) = a + b * x; % Fit function
	a = 0.01; b = 0.1; % Set reasonable starting values here
	set fit errorvariables;
	fit f(x) 'A4.txt' via a, b;
	plot [x=-90:90] f(x);
	set print 'parameters.dat'; % Open a file to save the parameters into
	print a, a_err; % Write the parameters to file
	print b, b_err;
};

\addplot [
	blue,
	only marks,
	mark = |,
	mark size=1.5,
	error bars/.cd,
	y dir = both,
	y fixed = 0.2,
	x dir = both,
	x fixed = 2 % 2° Abweichung
] table {A4.txt};

\addlegendentry{
	\pgfplotstableread{parameters.dat}\parameters % Open the file Gnuplot wrote
	\pgfplotstablegetelem{0}{0}\of\parameters \pgfmathsetmacro\paramA{\pgfplotsretval} % Get first element, save into \paramA
	\pgfplotstablegetelem{1}{0}\of\parameters \pgfmathsetmacro\paramB{\pgfplotsretval}
	$\pgfmathprintnumber{\paramB} \cdot \varphi
		+ \pgfmathprintnumber{\paramA}$
}

\end{axis}
\end{tikzpicture}
\caption{Winkelrichtgröße des großen Drehtellers bestimmen}\label{dia:A4}
\end{center}
\end{figure}

Die Funktion der Ausgleichsgerade für die Bestimmung der Winkelrichtgröße des großen Drehtellers lautet wie folgt:
\begin{equation}
F = \SI{0.06 \pm 0.06}{N} + \underbrace{\SI{0.1280 \pm 0.0011}{N}}_{a_4} \cdot \frac{180}{\pi} \cdot \varphi
\end{equation}
Analog zu Aufgabe 10 kann $D^*_g = \SI{2.127 \pm 0.024}{Nm}$ berechnet werden.

\subsection{Aufgabe 14}
Aus dem Volumen und der angegebenen Dichte des Zylinders lässt sich auf sein Trägheitsmoment schließen: $J_g = \frac{\pi h_g \rho R_g^4}{2} = \SI{0.69 \pm 0.04}{kgm^2}$

\subsection{Aufgabe 15}
$D^*_g$ lässt sich sowohl aus der Schwingungsdauer und dem Eigenträgheitsmoment als auch aus den statischen Drehmomenten bestimmen. Die dynamische Berechnung erfolgt nach: $D^*_g = \frac{4\pi^2}{T^2} \cdot J_g = \SI{2.41 \pm 0.13}{Nm}$ mit $T = \SI{3.36 \pm 0.07}{s}$

Dieser Wert ist deutlich größer und ungenauer als der Wert, der durch die statischen Messungen berechnet wurde, die Unsicherheitsbereiche überschneiden sich noch nicht einmal.

\subsection{Aufgabe 16}
Das Trägheitsmoment des Menschen kann Analog zu \cref{sub:A12} mit \cref{equ:TraegheitsmomentPuppe} berechnet werden. $J_{m1}$ ist das Trägheitsmoment mit angelegten Armen und $J_{m2}$ mit ausgestreckten Armen: $J_{m1} = \SI{0.89 \pm 0.09}{kgm^2}$ und $J_{m2} = \SI{2.14 \pm 0.018}{kgm^2}$
Als Verhältnis der Trägheitsmomente ergibt sich ca. $\frac{J_{m2}}{J_{m1}} = 2.4$.

\subsection{Aufgabe 17}
Die Masse des Menschen beträgt $\SI{67}{kg}$, sein Hüftumfang $\SI{83}{cm}$, seine Größe $\SI{1.82}{m}$, sein Armumfang $\SI{26}{cm}$, seine Armlänge $\SI{70}{cm}$, sein Beinumfang $\SI{42}{cm}$, seine Beinlänge $\SI{95}{cm}$, sein Kopfumfang $\SI{60}{cm}$ und seine Rumpfhöhe $\SI{60}{cm}$.
\begin{table}[ht]
\begin{center}\label{tab:Einzeltraegheitsmomente}
\begin{tabular}{r|ccc}\hline
\textbf{Körperteil} & Kopf & angelegte Arme ($2\times$) & ausgebreitete Arme ($2\times$) \\
\hline
\textbf{Massenanteil} [\%] & $7.3$ & $10.4$ & $10.4$ \\
\textbf{Trägheitsmoment} [kgm$^2$] & $0.018$ & $0.0030$ & $0.14$ \\
\textbf{Abstand} [cm] & $0$ & $17$ & $48$ \\
\hline\end{tabular}
\begin{tabular}{r|ccc}\hline
\textbf{Körperteil} & Beine ($2\times$) & Rumpf \\
\hline
\textbf{Massenanteil} [\%] & $33.4$ & $48.9$ \\
\textbf{Trägheitsmoment} [kgm$^2$] & $0.025$ & $0.14$ \\
\textbf{Abstand} [cm] & $10$ & $0$ \\
\hline\end{tabular}
\caption{Einzelträgheitsmomente des Menschen}
\end{center}
\end{table}

Der Gesamtträgheitsmoment des Menschen ist damit $J_{m1} = \SI{0.64}{kgm^2}$, bzw. $J_{m2} = \SI{2.3}{kgm^2}$, das Verhältnis der beiden etwa $\frac{J_{m2}}{J_{m1}} = 3.6$.

\subsection{Aufgabe 18}
Das Trägheitsmoment des Menschen kann auch aus dem Trägheitsmoment der Puppe extrapoliert werden. Dazu nähert man beide Formen mit Zylindern an und kommt zu folgender Formel:
\begin{equation}
J_m = \frac{m_m}{m_p} \cdot \left(\frac{h_m}{h_p}\right)^2 \cdot J_p
\end{equation}
Das ergibt $J_{m1} = \SI{0.85}{kgm^2}$ und $J_{m2} = \SI{3.3}{kgm^2}$.

\section{Diskussion}
Die berechneten Werte für $D^*_g$ aus Aufgabe 13 und 15 unterscheiden sich sehr stark. Dass sich die Unsicherheitsbereiche nicht überschneiden legt nahe, dass mindestens für einen der beiden Werte, größere Unsicherheiten existieren als angenommen wurden. Das ist vermutlich für letzteren Wert (dynamische Berechnung) der Fall. Zum einen gibt es nur drei Werte für die Zeit und zum anderen wurde $J_g$ rechnerisch aus einer Approximation des Drehtellers erstellt. Der Fehler dieser Annäherung ist vermutlich größer als durch die Unsicherheiten der Längen berechnet wurde. Z.B. wurde die Stange in der Mitte des Tellers vernachlässigt.

Interessant ist bei den Ergebnissen vor allem, dass durch das Ausstrecken der Arme etwa verdreifacht werden konnte. Im Umkehrschluss bedeutet das, dass man, wenn man sich mit ausgestreckten Armen dreht, seine Drehgeschwindigkeit durch Anziehen der Arme verdreifachen kann. Praktisch kann die Größenordnung dieses Ergebnisses leicht nachvollzogen werden.

\subsection{Aufgabe 19}
Die Ergebnisse für ausgestreckte Arme haben eine größere Differenz, wenn man den Menschen und die Puppe vergleicht. Dies lässt vermuten, dass die Puppe mehr Teilgewicht in den Armen hat, als dies beim Menschen der Fall war.\\
Eine mögliche Ursache für die allgemein kleineren praktischen Ergebnisse wäre eine engere Beinhaltung beim Menschen als bei der Puppe, die in beide Trägheitsmomente reduzierend einfließt.





\section{Weiterführende Fragen}

Eine Hängeschaukel ist ein Holzbrett das mittels 2 Seilen an einer Stange befestigt ist und frei schwingen kann.
Auf der Schaukel sitze ein Mensch.

\subsection{Energieformen}

Während der Schwingung treten zwei verschiedene Energieforman auf: kinetische und potentielle.
Am Endpunkt der Schwingung bleibt die Schaukel stehen, $E_kin$ ist null; $E_pot$ maximal.
Am unteren Durchgang andersherum.

\subsection{Drehimpuls}

Der Drehimpuls ist von der Geschwindigkeit und dem Drehmoment der Masse abhängig, daher ist er in den Umkehrpunkten null, wenn die Geschwindigkeit null ist, und am unteren Scheitelpunkt maximal.

\subsection{Beeinflussung}

Das Trägheitsmoment der Drehung ist von der Masse und dem Radius abhängig. Beim Absenken des Oberkörpers erhöht sich der durchschnittliche Radius der Masse und damit auch das Trägheitsmoment.
Dies ist auch ein wesentlicher Faktor beim aktiven schaukeln.

\subsection{Aktives Schaukeln}

Beim aktiven Schaukeln wird dem System durch die periodische Veränderung des Träg\-heits\-mo\-men\-tes Energie zugeführt.
Die Veränderung muss dabei mit $\pi \over 2$ periodischem Versatz durchgeführt werden.
Es muss also zum Zeitpunkt des größtmöglichen Drehimpulses das Träg\-heits\-mo\-men\-t verringert werden, um die Geschwindigkeit und damit die kinetische Energie (die Geschwindigkeit fließt in die Energie quadratisch ein, in den Drehimpuls aber nur linear) zu erhöhen.
So kommt es zum Resonanzfall.

\subsection{Bedeutung für den Drehimpuls}

Der maximale Drehimpuls erhöht sich ebenso wie der maximale Drehwinkel in jeder Schwingung, bis die Energieverluste durch Reibung, Verformung der Feder etc... die Energiezufuhr ausgleichen.


\section{Anhang}
\textbf{Fehlerrechnungen zu Aufgabe 10}

Die Unsicherheit der Steigung lässt sich mit linearer Regression berechnen. Dazu werden die Gleichungen (41), (42) und (44) aus dem Fehlerrechnungs-Skript verwendet. Der Fehler $\sigma_y$ wird nicht nach der angegebenen Formel berechnet, sondern es wird der vorher geschätzte Fehler der Kraft verwendet, der größer ist, um eine akuratere Abschätzung des Fehlers zu erhalten. Um die Ungenauigkeit der Winkelrichtgröße zu bestimmen wird mit quadratischer Addition die Ungenauigkeit des Radius hinzugerechnet.

\end{document}
