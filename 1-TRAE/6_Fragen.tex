


\section{Weiterführende Fragen}

Zur weiteren Überlegung wurde aufgegeben, zu folgendem Sachverhalt einige Fragen zu beantworten:\\
\emph{Eine Hängeschaukel ist ein Holzbrett das mittels 2 Seilen an einer Stange befestigt ist und frei schwingen kann.
Auf der Schaukel sitze ein Mensch.}

\subsection{Energieformen}

Während der Schwingung treten zwei verschiedene Energieforman auf: kinetische und potentielle.
Am Endpunkt der Schwingung bleibt die Schaukel stehen, $E_{kin}$ ist null; $E_{pot}$ maximal.
Am unteren Durchgang andersherum.

\subsection{Drehimpuls}

Der Drehimpuls ist von der Geschwindigkeit und dem Drehmoment der Masse abhängig, daher ist er in den Umkehrpunkten null, wenn die Geschwindigkeit null ist, und am unteren Scheitelpunkt maximal.

\subsection{Beeinflussung}

Das Trägheitsmoment der Drehung ist von der Masse und dem Radius abhängig. Beim Absenken des Oberkörpers erhöht sich der durchschnittliche Radius der Masse und damit auch das Trägheitsmoment.
Dies ist auch ein wesentlicher Faktor beim aktiven schaukeln.

\subsection{Energie bei Absenkung des Oberkörpers}

Nach dem Absenken des Oberkörpers bleibt die Energie des Systems, abgesehen von Reibungseffekten, konstant.
Je nach Zeitpunkt des Absenkens wird die Energie dabei erhöht (während der maximalen kinetischen Energie) oder nicht verändert (während der maximalen potentiellen Energie).

\subsection{Aktives Schaukeln}

Beim aktiven Schaukeln wird dem System durch die periodische Veränderung des Träg\-heits\-mo\-men\-tes Energie zugeführt.
Die Veränderung muss dabei mit $\pi \over 2$ periodischem Versatz durchgeführt werden.
Es muss also zum Zeitpunkt des größtmöglichen Drehimpulses das Träg\-heits\-mo\-men\-t verringert werden, um die Geschwindigkeit und damit die kinetische Energie (die Geschwindigkeit fließt in die Energie quadratisch ein, in den Drehimpuls aber nur linear) zu erhöhen.
So kommt es zum Resonanzfall.

\subsection{Bedeutung für den Drehimpuls}

Der maximale Drehimpuls erhöht sich ebenso wie der maximale Drehwinkel in jeder Schwingung, bis die Energieverluste durch Reibung, Verformung der Feder etc... die Energiezufuhr ausgleichen.
