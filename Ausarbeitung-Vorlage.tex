%&PDFLaTeX
\documentclass[a4paper,11pt]{article}
\usepackage{german}
\usepackage[utf8]{inputenc} % Eigentlich kann man damit dann auch Umlaute verwenden.
                            % Allerdings verstehen nicht alle Editoren utf8, daher
                            % sind die Umlaute in diesem Text mit Anf"uhrungszeichen.
\usepackage{fancyhdr}
\usepackage{graphicx}

\setlength{\parindent}{0ex}
\setlength{\parskip}{1.5ex plus0.5ex minus0.2ex}
\setlength{\headheight}{26pt}
\addtolength{\topmargin}{-0.2in}
\addtolength{\headsep}{0.2in}
\addtolength{\oddsidemargin}{-0.2in}
\addtolength{\textwidth}{0.5in}

\renewcommand{\topfraction}{1.0}
\renewcommand{\bottomfraction}{1.0}
\renewcommand{\textfraction}{0.0}

\newcommand{\stand}{08.12.08}
\pagestyle{fancy}
\rhead{\bf Stand: \stand \\ Seite \thepage}
\lhead{\bf Hinweise zur\\ Ausarbeitung}
\chead{\bf (AUS)}
\rfoot{}
\lfoot{}
\cfoot{}

\begin{document}
%\thispagestyle{empty}
\begin{center}
% Prinzipiell kann LaTeX auch Titelseiten erzeugen, ich habe mich
% hier trotzdem für die h"andische Variante entschieden.
\textbf{\Large Hinweise zum Schreiben einer Ausarbeitung}\\[1ex]
\textbf{\large (Physikalisches Praktikum f\"ur Physiker)}\\[5ex]
       Autor: M. Sa"s\\
       Fakult"at f"ur Physik\\
       Technische Universit"at M"unchen\\[1ex]
       \stand
\end{center}

\tableofcontents
\clearpage

Machen Sie sich einmal die M"uhe, eine eine vern"unftig strukturierte Ausarbeitung als 
\glqq Vorlage\grqq\ zu erstellen, die Sie dann durch das ganze Praktikum (evtl. sogar
das Fortgeschrittenenpraktikum) verwenden k"onnen.

Formulieren Sie die Ausarbeitung in ganzen S"atzen. Eine Stichpunktsammlung
mit Rechnungen ist nicht ausreichend. Die L"ange der Ausarbeitung darf aber 10 
Seiten nicht "uberschreiten. 
Was dar"uber hinaus geht, wird nicht gewertet!

\section{Struktur einer Ausarbeitung}\label{sec:struktur}
Der typische Aufbau einer wissenschaftlichen Arbeit besteht aus:
\begin{enumerate}
\item Einleitung\\[-1.7em] % Der Abstand zwischen den Aufz"ahlungspunkten war mir hier zu gross.
\item Beschreibung der verwendeten Methoden\\[-1.7em]
\item Experimentelles Vorgehen\\[-1.7em]
\item Ergebnisse (mit Betrachtung der Unsicherheiten)\\[-1.7em]
\item Diskussion der Ergebnisse\\[-1.7em]
\item Zusammenfassung\\[-1.7em]
\item Anhang (wenn nötig)\\[-1.7em]
\item Literaturverzeichnis\\
\end{enumerate}
An diesem Aufbau sollten Sie sich orientieren, wenn Sie Ihre Ausarbeitung
schreiben. Die Grenzen zwischen den einzelnen Bereichen können dabei
durchaus variieren. Es kann zweckmäßig sein, einzelne Punkte zusammenzufassen
(z.B. Methoden und experimentelles Vorgehen, Ergebnisse und Diskussion,
Diskussion und Zusammenfassung).

Bei manchen Versuchen, bei denen verschiedene Methoden verwendet werden,
lassen sich auch die Bereiche 2-5 für jede Methode einzeln ausführen.

Der Aufbau muss aber so sein, dass klar nachvollziehbar ist, was Sie
gemacht haben, und welcher Teil der Ausarbeitung zu welchem Teil
des Versuches gehört.

\subsection{Die Einleitung}
  In der Einleitung soll die Idee des Versuches kurz wiedergegeben werden.
  Sie soll h"ochstens zwei bis drei S"atze umfassen, au"ser es werden darin 
  die zu manchen Versuchen vorhandenen  \glqq Vor"uberlegungen\grqq\ behandelt.
\subsection{Die Beschreibung der verwendeten Methoden}
  Hier wird die verwendete Methode kurz skizziert. Dabei sollen die für die 
  Auswertung wichtigen Formeln explizit angegeben werden.
  Die Beschreibung muss aber nicht so detailliert sein, wie die Versuchsanleitung,
  es ist sinnvoller, auf die Versuchsanleitung zu verweisen (s. Literaturverzeichnis).
  Allerdings müssen alle Größen, die gemessen oder verwendet werden, benannt sein.
\subsection{Das Experimentelle Vorgehen}
  In diesem Punkt wird beschrieben, wie tatsächlich gemessen wurde, z.B. wie viele
  Messungen gemacht wurden, oder mit welcher Apparatur und welchen Geräten Sie gemessen 
  haben (falls vorhanden Ger"atenummer, o."a.).  
\subsection{Die Ergebnisse}
   Geben Sie Ihre Messergebnisse (nicht jedoch alle Einzelmessungen) mit Unsicherheiten an,
          und f"uhren auf, welche Unsicherheiten ber"ucksichtigt wurden, und welche nicht.

   Stellen Sie Ihre Messdaten anschaulich in Graphen oder kurzen Tabellen dar, 
          und beschreiben diese kurz.

   Abschlie"send wird das aus den Messergebnissen gewonnene Endergebnis mit Unsicherheit angegeben.

   Die ausf"uhrliche Fehlerrechnung und l"angere Zwischenrechnungen sind meist besser in 
          einem Anhang als im Text aufgehoben.
\subsection{Die Diskussion}
   Diskutieren Sie Ihr Ergebnis: Was sagt es aus? Ist es realistisch?

   Falls dieselbe Gr"o"se auf unterschiedliche Arten bestimmt wurde, m"ussen die 
   Ergebnisse vergleichend diskutiert werden.

   Wenn m"oglich ist das Ergebnis mit Literaturwerten zu vergleichen (Zitate nicht vergessen!)
   Mögliche Gründe für eventuelle Abweichungen müssen angeführt werden.
\subsection{Die Zusammenfassung}
   Fassen Sie die wichtigsten Punkte des Versuchs noch einmal zusammen.
   
\subsection{Der Anhang}
   In einem Anhang kann all das dargestellt werden, was für den Textfluss
   der eigentlichen Ausarbeitung zu lang ist. Dies sind im Praktikum
   hauptsächlich längere Rechnungen, z.B. eine ausführliche Fehlerrechnung.
   Außerdem kann die Beantwortung der Fragen zu den Versuchen im 
   Anhang erfolgen. % (nur Semesterpraktikum).

   Die Rohdaten der Messungen können im Anhang (und nicht
   im Text) tabelliert werden. Dies ist hilfreich, wenn das Protokoll nur
   schwer lesbar ist (sollte eigentlich nicht vorkommen!), oder um die Versuche
   \glqq komplett\grqq\ zu archivieren. 
\subsection{Das Literaturverzeichnis}
   Wenn Sie Informationen verwenden, die nicht von Ihnen selbst stammen, müssen
   Sie die Quelle zitieren. Im Praktikum wird man meist zumindest die Anleitung als
   Referenz angeben. Aber auch bei Literaturwerten, mit denen die gemessenen Werte
   verglichen werden, muss die Herkunft angegeben werden.
   
   Die Referenz auf die vorliegenden Hinweise kann z.B. so aussehen:
   \begin{enumerate}
   \item M.~Saß, \glqq Hinweise zum Schreiben einer Ausarbeitung\grqq, \stand \\
         Bezugsquelle:\\
         {\small http://www.ph.tum.de/studium/betrieb/praktika/anfaenger/anleitungen/AUS.pdf} 
   \end{enumerate}

\section{Was muss in die Anleitung?}
%Folgende Punkte m"ussen (zwingend) in der Ausarbeitung enthalten sein:
\begin{enumerate}
\item \textbf{Namen und Gruppennummer}\\[-2em]
\item \textbf{Titel des Experiments}\\[-2em]
\item \textbf{Datum der Durchf"uhrung}\\[-2em]
\item \textbf{Inhaltsverzeichnis} (nicht zwingend, aber sinnvoll)\newline
  Es erleichtert (sowohl Ihnen als auch dem Korrektor) zu überprüfen, 
  ob die Ausarbeitung vollständig ist und eine vernünftige Struktur hat.
\item[]\hspace*{-1.8em} Diese ersten vier Punkte können z.B. auf einer 
  Titelseite zusammengefasst sein.

\item \textbf{Seitennummerierung}

\item \textbf{Experimentbeschreibung}\\[-2em]
\item \textbf{Ergebnisse mit Unsicherheiten}\\[-2em]
\item \textbf{Diskussion der Ergebnisse}\\[-2em]
\item \textbf{Referenzen}
\item[]\hspace*{-1.8em} Die letzten vier Punkte betreffen die
   Struktur der Ausarbeitung (s.~Abschnitt~\ref{sec:struktur}).
\end{enumerate}


\section{Was gilt es zu vermeiden?}
\begin{enumerate}
\item L"angere Messwerttabellen geh"oren (wenn "uberhaupt) nur in einen Anhang.
\item L"angere Rechnungen oder Umformungen sollten im Anhang beigef"ugt werden.
      Dies gilt auch f"ur eine \emph{ausf"uhrliche} Fehlerrechnung. 
\item Die Angabe von Ergebnissen mit zu vielen Stellen ist nicht nur
      unschön, sondern falsch.\newline 
      Wert und Unsicherheit sollen die gleiche Genauigkeit haben. Die Angabe
      der Unsicherheit auf mehr als zwei signifikante Stellen ist im Normalfall
      nicht sinnvoll.
\item Offensichtlich falsche Ergenisse dürfen nicht einfach so stehen gelassen werden.
      Meistens hilft ein nochmaliges Nachrechnen. Falls der Fehler auch dann nicht
      aufzufinden ist, muss das Ergebnis mindestens kommentiert werden (s.~Diskussion).
\end{enumerate}


\section{Formattierungsfragen}
\subsection{Text und Überschriften}
Wählen Sie eine gut lesbare Schriftgröße (10-12~pt), die Sie für den 
ganzen Text beibehalten.%, und einen  Zeilenabstand von etwa 1,5. 

Überschriften und Unterüberschriften sollen nummeriert,
und ihrer Hierarchie entsprechend hervorgehoben werden (Größe, Fettdruck, etc.)

\subsection{Formeln}
Formeln und Gleichungen werden in eigenen Zeilen gesetzt und nummeriert. 
So kann man sp"ater im Text darauf verweisen. 
\begin{center}
\parbox{7cm}{\small
Dies ist ein Beispiel, wie man eine 
Formel in der Ausarbeitung setzt.
\begin{equation}\label{Linsengleichung}
\frac{f}{a} + \frac{f'}{a'} = 1
\end{equation} 
Die Gleichung~(\ref{Linsengleichung}) stellt die
allgemeine Form der Linsengleichung dar.}
\end{center}

\subsection{Tabellen}
Tabellen werden durchnummeriert, so kann im  Text  darauf  verwiesen werden.
Tabellen sollen mit einer  Titelzeile und Spalten"uberschriften versehen sein.
Daten werden entweder im Text oder in Tabellen gezeigt, aber nicht in beidem.

Tabelle~\ref{tab:Beispiel} ist ein Beispiel, wie Tabellen formatiert 
werden können.
\begin{table}
\begin{center}
\begin{tabular}{cccc}
\hline
\textbf{Frequenz} & \textbf{Drehmoment} 
           & \textbf{Leistung} & \textbf{Wirkungsgrad} \\ 
\textbf{$f$ (Hz)} & \textbf{$M$ (Nm)} 
           & \textbf{$P_\mathrm{m}$ (W)} & \textbf{$\eta$ (\%)} \\ \hline 
$4,10\pm 0,20$ & $0,176\pm 0,017$ & $4,52\pm 0,50$ & $3,48\pm 0,56$ \\
$4,50\pm 0,20$ & $0,153\pm 0,015$ & $4,32\pm 0,47$ & $3,33\pm 0,53$ \\
$5,00\pm 0,20$ & $0,117\pm 0,012$ & $3,67\pm 0,40$ & $2,83\pm 0,45$ \\
$5,40\pm 0,20$ & $0,059\pm 0,007$ & $1,98\pm 0,25$ & $1,53\pm 0,27$ \\
$5,80\pm 0,20$ & $0,025\pm 0,005$ & $0,90\pm 0,19$ & $0,69\pm 0,17$ \\ \hline
\end{tabular}
\caption{\label{tab:Beispiel}Kenndaten des Stirlingmotors bei einer Heizleistung 
    $P_\mathrm{h}=130\pm 15$~W: Drehmoment $M$, abgegebene mechanische Leistung
    $P_\mathrm{m}$ und Wirkungsgrad $\eta$ bei der Frequenz $f$}
\end{center}
\end{table} 



\subsection{Abbildungen}
Abbildung~\ref{abb:beispiel} ist ein Beispiel für die Formatierung
einer Abbildung.
\begin{figure}[t]
\begin{center}
%\includegraphics*[angle=270, width=11cm]{beispielbild}
\caption{Frequenzabhängigkeit des Drehmoments der Stirlingmaschine.
Die gestrichelte Linie dient der Augenführung}\label{abb:beispiel}
\end{center}
\end{figure} 

Abbildungen werden nummeriert und durch eine Bildunterschrift oder
Legende kurz erläutert. Im Text muss Bezug auf die Abbildung 
genommen werden, dort werden die Abbildungen beschrieben. 

Die Achsen eines Graphen m"ussen beschriftet sein (aufgetragene Gr"o"se mit Einheit).
Zumindest einige Datenpunkte sollen mit Fehlerbalken versehen werden.

Die gezeigten Daten sollen die Fl"ache eines Graphen ausnutzen. Die Achsen d"urfen 
"uber den Wertebereich der Daten nicht zu weit hinausgehen. Es ist nicht
erforderlich, dass der Ursprung des Koordinatensystems immer in der Graphik sichtbar ist.

\section{Was ist sonst zu beachten?}
\begin{enumerate}
\item Alle Gr"o"sen m"ussen  mit Einheiten angegeben werden (es sei denn, sie sind dimensionslos).\\
      Nach M"oglichkeit sind SI-Einheiten zu verwenden (oder die im jeweiligen Fachgebiet 
      "ubliche Nomenklatur).
      Die Einheit wird "ubelicherweise mit einem Leerzeichen von der Zahl abgesetzt 
      und nicht kursiv gesetzt (z.B. $3,0$~V). 
\item Es sind geeignete \glqq Vorsilben\grqq\ ($\mu$, m, k, M, ...) zu verwenden 
      (z.B. $3,0$~mm, und nicht $0,0030$~m).
\item Gr"o"se und Unsicherheit sind mit der gleichen Genauigkeit und Einheit anzugeben.
      (z.B. $(3,00 \pm 0,04)$~m, und nicht $3~\mathrm{m}\pm 4~\mathrm{cm}$).\newline
      Ist die führende Ziffer der Unsicherheit eine Eins oder Zwei wird die Unsicherheit
      auf zwei signifikante Stellen angegeben, sonst reicht auch eine Stelle.
\item Es ist "ublich, in der Experimentbeschreibung eine personalisierte
      Formulierung zu vermeiden (Also: \glqq Es wurde gemessen\grqq\  
      statt \glqq Wir haben gemessen\grqq)
\end{enumerate}

\end{document}
